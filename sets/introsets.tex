%Read the slack post (link below) regarding syntax and formatting before you start writing lecture notes.
% Post: https://musa-2021.slack.com/archives/C01DGR645SL/p1609187728029500

\documentclass[../main.tex]{subfiles}
\begin{document}

\section{Week 4: Sets and Set Operations}
Together with propositional logic, set theory underpins the vast majority of modern mathematics, and is a crucial component of almost any proof you will encounter in any upper division course. The language of set theory provides us with a framework to formalize many mathematical concepts that we are intuitively familiar with---namely, collections of data and functions between them. Throughout the remainder of this course, and almost certainly beyond it, set theory will be used extensively.

\subsection{Sets}
We begin with the definition of a set.
\begin{definition}[set, element]
    A \dfn{set} $X$ is a collection of objects. An object $x$ in a set $X$ is called an \dfn{element} or \emph{member}. If $x$ is an element of $X$, then we write $x \in X$. Similarly, if $x$ is not an element of $X$, then we write $x \notin X$. Two sets are equal if and only if they have the same elements.
\end{definition}
\begin{example}[empty set]
    There is a set, denoted $\emp$, which contains no elements. We call $\emp$ the \dfn{empty set}.
\end{example}
Note that $X = \{1, 1, 1, 1\}$ is the same set as $Y = \{1\}$. After all, $1 \in X$, and $1$ is the only number with this property. So sets don't recognize multiple ``copies" of their elements. Sets also do not respect order; for example, $\{1, 2, 3\} = \{3, 2, 1\}$.

We will also often want to talk about a set contained within some other, larger set.
\begin{definition}[subset]
    Let $X$ and $Y$ be sets. If $y\in Y$ implies $y\in X$ for all $y$, then we say that $Y$ is a \dfn{subset} of $X$, and we write $Y \subseteq X.$ If also $Y \neq X$, we write $Y \subsetneq X,$ and we say that $Y$ is a \dfn{proper subset} of $X$.
\end{definition}
\begin{warn}
    Some authors will use $Y \subset X$ to mean either that $Y$ is a subset or a proper subset of $X$! While both conventions are acceptable, it's best to choose one of the two and be as consistent with this choice as possible in your writing; to avoid ambiguity, it also helps to explicitly state when a subset is proper.
\end{warn}
For example, Barack Obama (let's denot him $O$) is an element of the set $P$ of all presidents of the United States, so we can write $O\in P$. To write down all the elements of $P$, we can say
\[P = \{\text{Joe Biden}, \text{Donald Trump}, \text{Barack Obama}, \text{George W. Bush}, \dots\}.\]
If $Q$ denotes the set of all world leaders, then $P \subset Q$. For example, $O \in Q$. Is $P \in Q$? No, because the set of all presidents is not a world leader.

Next, let's define some special sets, written in blackboard font to emphasize their importance.
\begin{definition}
    The following sets will be used throughout your mathematical career.
    \begin{itemize}
        \item $\NN$ is the set of all natural numbers: $\NN \coloneqq \{0, 1, 2, \dots\}$.
        \item $\ZZ$ is the set of all integers: $\ZZ \coloneqq \NN \cup \{-1, -2, \dots\}$.
        \item $\QQ$ is the set of all rational numbers.
        \item $\RR$ is the set of all real numbers.
        \item $\CC$ is the set of all complex numbers.
    \end{itemize}
\end{definition}
\begin{warn}
    Some authors use $\NN$ to refer to the set of positive integers $\{1,2,3,\ldots\}$. We will use the notation $\ZZ^+$ to refer to this set.
\end{warn}
Notice that
\[\NN \subsetneq \ZZ \subsetneq \QQ \subsetneq \RR \subsetneq \CC.\]
Also note that some mathematicians exclude 0 from $\NN$. If it matters whether $0 \in \NN$, we'll try to indicate whether it's true or not.

\subsection{Set Operations}
Now, we will define several useful sets and operations on them, many of which you will encounter in almost any upper division math class.

The simplest operations involve two sets, so we begin with those.
We begin with operations which act on single sets.
\begin{definition}[union, intersection, product]
    Let $X$ and $Y$ be sets.
    \begin{itemize}
        \item The \dfn{union} of $X$ and $Y$, written $X \cup Y$, is the set consisting of elements in $X$ or $Y$:
        \[X \cup Y \coloneqq \{z: z \in X \text{ or } z \in Y\}.\]
        \item The \dfn{intersection} of $X$ and $Y$, written $X \cap Y$, is the set consisting of elements in $X$ and $Y$:
        \[X \cap Y \coloneqq \{z: z \in X \text{ and } z \in Y\}.\]
        \item The \dfn{product} of $X$ and $Y$, written $X \times Y$, is the set of all ordered pairs of elements in $X$ and in $Y$:
        \[X \times Y \coloneqq \{(x, y): x \in X \text { and } y \in Y\}.\]
    \end{itemize}
\end{definition}
With that said, there are operations on single sets.
\begin{definition}[power set, union, intersection]
    Let $X$ be a set.
    \begin{itemize}
        \item The \dfn{power set} of $X$, written $\mathcal P(X)$ or $2^X$, is the set of all subsets of $X$:
        \[\mathcal P(X) \coloneqq \{Y: Y \text{ is a set and } Y \subseteq X\}.\]
        \item If $\mc F=X$ is a ``collection'' or ``family'' of sets (i.e., a set containing sets), then the \emph{union} of all the sets in $\mathcal F$, written $\bigcup \mathcal F$, is
        \[\bigcup \mathcal F \coloneqq \{z: \text{there is a set } Z \in \mathcal F \text{ such that } z \in Z\}.\]
        \item If $\mc F=X$ is a collection of sets, then the \emph{intersection} of all the sets in $\mathcal F$, written $\bigcap \mathcal F$, is
        \[\bigcap \mathcal F \coloneqq \{z: z\in Z\text{ for all sets } Z \in \mathcal F\}.\]
    \end{itemize}
\end{definition}
\begin{example}
    The intersection of the set $P$ of presidents of the United States and the set $R$ of royalty of the United Kingdom is empty: $P \cap R = \varnothing$. Their product $P \times R$ consists of all the different ways we could pair a president with a royal; for example, $(\text{Abraham Lincoln},\text{Queen Elizabeth I})\in P \times R$.
\end{example}
\begin{remark}
    An ordered pair is not the same thing as a set with two elements: ordered pairs allows for repetition, and order matters.
\end{remark}
% For example,   Note that  So this element should not be confused with the set $\{\text{Abraham Lincoln},\text{Queen Elizabeth I}\}$.
Here are a few exercises for you to try.
\begin{exercise}
    Explain why it's true that every element of $\varnothing$ is even. Is it also true that every element of $\varnothing$ is odd? 
\end{exercise}
\begin{exercise}
    Suppose $A$, $B$, $C$, and $D$ are sets. Is it true that $(A \times B) \cup (C \times D) = (A \cup C) \times (B \cup D)$? If yes, prove this claim; if not, find a counterexample, and decide whether one of these sets contains the other.
\end{exercise}
\begin{exercise}
    Let $X$ be the set $\{1, 2, 3\}$. Determine the following.
    \begin{enumerate}[label=(\alph*)]
        \item What is the power set $\mathcal{P}(X)$?
        \item Is $1$ an element of $\mathcal{P}(X)$? What about $\{1\}$?
        \item Is $\{2, 3\}$ a subset of $\mathcal{P}(X)$? What about $\{\{2, 3\}\}$? What about $\{\{2\}, \{3\}\}$?
        \item Is $\varnothing$ an element of $\mathcal{P}(X)$? Is $\varnothing$ a subset of $\mathcal{P}(X)$?
        \item Is $X$ an element of $\mathcal{P}(X)$? Is $X$ a subset of $\mathcal{P}(X)$?
    \end{enumerate}
\end{exercise}
Let's give a few examples of theorems about sets.
\begin{theorem} \label{thm:px-is-poset}
    Let $X$, $Y$, and $Z$ be sets. Then the following are true.
    \begin{enumerate}[label=(\alph*)]
        \item $X \subseteq X$.
        \item If $X \subseteq Y$ and $Y \subseteq X$ then $X = Y$.
        \item If $X \subseteq Y$ and $Y \subseteq Z$ then $X \subseteq Z$.
    \end{enumerate}
\end{theorem}
\begin{proof}
    We show these one at a time.
    \begin{enumerate}[label=(\alph*)]
        \item For every $x \in X$, we see $x \in X$. So $X \subseteq X$.
        \item Assume $X \subseteq Y$ and $Y \subseteq X$. Then $x \in X$ if and only if $x \in Y$, so $X=Y$ follows.
        \item Assume $X \subseteq Y$ and $Y \subseteq Z$. Let $x \in X$; we must show $x \in Z$. Because $x \in X$, it follows that $x \in Y$. Because $x \in Y$, it follows that $x \in Z$.
        \qedhere
    \end{enumerate}
\end{proof}
\begin{remark}
    When we study relations in \cref{sec:relations}, we will see that \autoref{thm:px-is-poset} implies that the relation $\subseteq$ is a ``partial order'' of $\mc P(X)$. One also says that $\pset(X)$ is a ``poset,'' for ``partially ordered set.''
\end{remark}
% Any \emph{binary relation} (symbol you can write between a pair of elements of a set --- we will discuss relations in greater detail in chapter 4) that has these properties is called a \dfn{partial ordering}. So we have just proved that for any set $X$, $\subseteq$ is a partial ordering of $\pset(X)$. ()
Here is the last operation of this section, requiring two or three sets depending on viewpoint.
\begin{definition}[complement]
    Suppose $A$ and $B$ are sets, both contained in a set $X$. The \emph{complement} of $A$ in $X$, denoted $A^c$, is the set
    $$A^c = \{x \in X: x \notin A\}$$
    Similarly, write $A - B$ or $A \setminus B$ for the \emph{relative complement} (or \emph{set difference}) of $A$ with $B$,
    $$A \setminus B = \{x \in A : x \notin B\}$$
\end{definition}
\begin{exercise}
    If $A$ is a subset of a set $X$, express $A^c$ as a relative complement.
\end{exercise}
\begin{exercise}
    Let $A$ and $B$ be subsets of a set $X$ Prove the following.
    \begin{enumerate}[label=(\alph*)]
        \item $A \subseteq A \cup B$.
        \item $A \cap B \subseteq B$.
        \item $A \setminus B \subseteq A$.
    \end{enumerate}
\end{exercise}
\begin{theorem}[de Morgan's laws] \nirindex{de Morgan's laws} \label{thm:de-morgan}
    Suppose that $A$, $B$, and $C$ are subsets of $X$. Then the following are true.
    \begin{enumerate}[label=(\alph*)]
        \item $(A^c)^c = A$.
        \item $(A\cap B)^c = A^c \cup B^c$.
        \item $(A\cup B)^c = A^c \cap B^c$.
    \end{enumerate}
\end{theorem}
\begin{proof}
    Generally speaking, to show that two sets are equal, it suffices by \Cref{thm:px-is-poset} to show that they are subsets of each other. We will use this approach for (b) and (c).
    \begin{enumerate}[label=(\alph*)]
        \item By definition, $x \in (A^c)^c$ if and only if $x$ is not in $A^c$. But $A^c$ consists of precisely those elements of $X$ not in $A$, so $x \notin A^c$ if and only if $x \in A$. In total,
        \[x\in(A^c)^c\text{ if and only if }x\in A,\]
        so $A=(A^c)^c$ follows.
        
        \item In one direction, let $x \in (A \cap B)^c$, and we show $x\in A^c\cup B^c$. Then $x \notin A \cap B$. If $x \in A$ and $x \in B$, then this is a contradiction, so $x \notin A$ or $x \notin B$. So $x \in A^c \cup B^c$. Therefore, $(A \cap B)^c \subseteq A^c \cup B^c$.
        
        For the other direction, let $x \in A^c \cup B^c$, and we show $x\in(A\cap B)^c$. Well, either $x \notin A$ or $x \notin B$, so $x \notin A \cap B$, so $x \in (A \cap B)^c$. Therefore, $(A \cap B)^c \supseteq A^c \cup B^c$.
        
        \item This proof is similar to (b). In one direction, let $x \in (A \cup B)^c$. Thus, $x\notin A\cup B$, so $x\notin A$ or $x\notin B$. It follows $x\in A^c\cap B^c$ and so $(A\cup B)^c\subseteq A^c\cap B^c$.
        
        In the other direction, let $x\in A^c\cup B^c$. Then $x\notin A$ and $x\notin B$, so $x\notin A\cup B$. It follows $x\in(A\cup B)^c$ and so $(A^c\cup B^c)\subseteq(A\cup B)^c$.
        \qedhere
    \end{enumerate}
\end{proof}

\subsection{Problems}
\begin{homework}
    Let $X$, $Y$, and $Z$ be sets. Show the following.
    \begin{enumerate}[label=(\alph*)]
        \item $X \cap (Y \cup Z) = (X \cap Y) \cup (X \cap Z)$.
        \item $X \cup (Y \cap Z) = (X \cup Y) \cap (X \cup Z)$.
    \end{enumerate}
    (Hint: Try drawing a Venn diagram to visualize each set!)
\end{homework}
\begin{homework}
    Suppose $A, B, C$ are subsets of $X$. Write $A \operatorname{\triangle} B$ for the \dfn{symmetric difference} of $A$ and $B$ in $X$, which is
    $$A \operatorname{\triangle} B = (A \setminus B) \cup (B \setminus A).$$
    In the Venn diagram of $A$ and $B$, $A \operatorname{\triangle} B$ consists of the parts of the diagram that are in exactly one of $A$ and $B$, but not both.
    
    Show the following.
    \begin{enumerate}[label=(\alph*)]
        \item $x \in X$ has $x \in A \operatorname{\triangle} B$ if and only if $x \in A$ or $x \in B$ but not both.
        \item $(A \operatorname{\triangle} B) \operatorname{\triangle} {(B \operatorname{\triangle} C)} = A \operatorname{\triangle} C$. (Hint: break each step into cases and apply part 1)
        %Removed part 3 because this problem is already very long, and the last part is not very interesting
        %\item $A \cap (B \Delta C) = (A \cap B) \Delta (A \cap C)$.
        \item $(A \operatorname{\triangle} B)\operatorname{\triangle}C=A\operatorname{\triangle}{}(B\operatorname{\triangle}C)$.
        \item $A\cap(B\operatorname{\triangle}C)=(A\cap B)\operatorname{\triangle}{}(A\cap C)$.
    \end{enumerate}
\end{homework}
\begin{homework}
    Suppose $X$ is some set.
    \begin{enumerate}[label=(\alph*)]
        \item Find a subset $E_1\subseteq X$ such that for every $A \subseteq X$, we have $E_1 \cup A = A$. Is $E_1$ unique?
        \item Find a subset $E_2\subseteq X$ such that for every $A \subseteq X$, we have $E_2 \cap A = A$. Is $E_2$ unique?
        \item Find a subset $E_3\subseteq X$ such that for every $A \subseteq X$, we have $A \setminus E_3 = A$. Is $E_3$ unique?
    \end{enumerate}
\end{homework}
\begin{homework}
    Let $X$ be a set. Show that $\bigcup \mathcal P(X) = X$ and $\bigcap \mathcal P(X) = \varnothing$.
\end{homework}

\end{document}