%Read the slack post (link below) regarding syntax and formatting before you start writing lecture notes.
% Post: https://musa-2021.slack.com/archives/C01DGR645SL/p1609187728029500



\documentclass[../main.tex]{subfiles}
\begin{document}

\section{Week 8: Strong Induction and Well Ordering}
We've already seen how induction can serve as a powerful proof-writing tool under the right conditions. We will now look at how it can be generalized to attack a broader class of problem, allowing us to perform induction with fewer constraints and over a wider variety of sets.

\subsection{Strong Induction}
First, we will turn our attention to ``strong induction,'' which looks very similar to the induction we've already been introduced to. In fact, strong induction is equivalent to ``regular'' induction, as we will see in \Cref{thm:well-order-equivs}.
\begin{theorem}[Principle of strong induction]
\index{principle of induction}
    Let $P$ be a statement indexed by $\NN$. To show $P(n)$ is true for all $n \in \NN$, it suffices to show the following.
    \begin{enumerate}
        \item Base case: $P(0), P(1) \dots, P(m)$ are true for some nonnegative integer $m$.
        \item Inductive step: Let $k \in \NN$. If $P(\ell)$ is true for every $\ell < k$, then $P(k)$ is true.
    \end{enumerate}
\end{theorem}
The important difference between induction and strong induction is that the former has a single base case and only advances one step at a time. On the other hand, strong induction allows you to assume all previous cases are true, which is often necessary to prove certain results, several of which we will explore subsequently. We defer a formal proof of \Cref{thm:well-order-equivs}, but we will provide an explanation similar to what we did for \Cref{thm:induction}.
\begin{proof}[Explanation]
    By the case, we are given that $P(0),P(1),\ldots,P(m)$ are true for some nonnegative integer $m$. To show $P(m+1)$, we see that $P(\ell)$ is true for $\ell<m+1$ already, so the inductive step finishes. Similarly, to show $P(m+2)$, we now know that $P(\ell)$ is true for all $\ell<m+2$ (including $m+1$), so the inductive step finishes. This process is able to show $P(n)$ is true for all $n\in\NN$ eventually.
\end{proof}
Observe that, like ``regular'' induction, our base case need not start at $0$. The explanation that we could start with any consecutive $m + 1$ integers rather than $0, \dots, m$ is identical to what we did above.
\begin{example}
    Define the Fibonacci sequence $F_0,F_1,F_2,\ldots$ by $F_n=F_{n-1}+F_{n-2}$ and $F_0=0$ and $F_1=$. For each $n \in \NN$, we have
    \[F_n = \frac{\varphi^n - (1 - \varphi)^n}{\sqrt{5}},\]
    where $\varphi = \frac{1 + \sqrt{5}}{2}$.
\end{example}
\begin{proof}
    We proceed by strong induction on $n$, using $n = 0, 1$ as our base cases.
    \begin{enumerate}
        \item Base case: note $F_0=0$ and $\varphi^0 = (1 - \varphi)^n = 1$, so the claim holds for $n=0$. For $n=1$, we compute
        \[\frac{\varphi-(1-\varphi)}{\sqrt5}=\frac{\frac{1+\sqrt5}2-\frac{1-\sqrt5}2}{\sqrt5}=1,\]
        which is indeed $F_1$.
        \item Inductive step: let $k \in \NN$ be greater than 1, and suppose we have already shown that
        \[F_\ell = \frac{\varphi^\ell - ( 1 - \varphi)^\ell}{\sqrt{5}}\]
        for each $\ell < k$. (In practice, we will only need $\ell=k-1$ and $\ell=k-2$.) By definition, we have
        \[F_k = F_{k - 1} + F_{k - 2},\]
        so from our inductive hypothesis we obtain
        \begin{equation}
            F_k = \frac{\varphi^{k - 1} - (1 - \varphi)^{k - 1}}{\sqrt{5}} + \frac{\varphi^{k - 2} - (1 - \varphi)^{k - 2}}{\sqrt{5}}. \label{eq:almost-done-fib}
        \end{equation}
        This doesn't look great, but we promise that all that remains is some algebraic manipulation.
        
        We could try to expand the binomial in the expression above and then try to cancel terms, but that would be messy and tedious. Instead, it helps to make the observation that $\varphi=\frac{1+\sqrt5}2$ and $1 - \varphi=\frac{1-\sqrt5}2$ are the conjugate roots of the polynomial $x^2 - x - 1$. (This an be verified by direct computation or the quadratic formula.) In other words,
        \begin{align*}
            \varphi^2 &= \varphi + 1,\\
            (1 - \varphi)^2 &= (1 - \varphi) + 1.
        \end{align*}
        Multiplying both sides of the first equation by $\varphi^{k - 2}$, we obtain
        \begin{equation}
            \varphi^k = \varphi^{k - 1} + \varphi^{k - 2}. \label{eq:golden-ratio}
        \end{equation}
        Similarly, we have
        \begin{equation}
            (1 - \varphi)^k = (1 - \varphi)^{k - 1} + (1 - \varphi)^{k - 2}. \label{eq:alt-golden-ratio}
        \end{equation}
        Substituting \autoref{eq:golden-ratio} and \autoref{eq:alt-golden-ratio} into \autoref{eq:almost-done-fib}, we have
        \[F_k = \frac{\varphi^k - (1 - \varphi)^k}{\sqrt{5}},\]
        which completes the proof of the inductive step.
        \qedhere
    \end{enumerate}
\end{proof}
Note that regular induction could not have proven the statement in the previous example without modifications. This is because in order to prove a claim about $F_n$, we actually need to look at the previous two iterations of the Fibonacci sequence, namely $F_{n-1}$ and $F_{n - 2}$. Regular induction does not give us the ability to do this, because we proved only a single base case: we would run into trouble as soon as we looked at $F_2 = F_1 + F_0$.

However, with that said, it is possible to make some modifications in order for regular induction to go through. The idea here is to add base cases by hand to the assertion we're trying to prove. Here's what that looks like concretely.
\begin{exe} \label{exe:binet-regular-induction}
    For example, for $n\in\NN$, let $P(n)$ denote the assertion
    \[F_n=\frac{\varphi^n-(1-\varphi)^n}{\sqrt5}\qquad\text{and}\qquad F_{n+1}=\frac{\varphi^{n+1}-(1-\varphi)^{n+1}}{\sqrt5}.\]
    (Namely, $P(n)$ has two equalities.) Show $P(n)$ is true for all $n\in\NN$ using regular induction.
\end{exe}
As another application of strong induction, we prove part of the Fundamental theorem of arithmetic. To do so, we define prime factorizations.
\begin{definition}[prime] \nirindex{prime}
    A positive integer $n\in\NN$ is \dfn{prime} if and only if $n>1$ and $n$ cannot be written as $n=ab$ for positive integers $a$ and $b$ greater than $1$.
\end{definition}
\begin{remark}
    Some authors prefer the word ``irreducible'' to ``prime'' in the above definition.
\end{remark}
\begin{definition}[prime factorization]
    If $n \in \NN$ is a natural number, a \dfn{prime factorization} of $n$ is a product of the form
    \[n = p_1\cdot p_2\cdot\ldots\cdot p_m,\]
    where each $p_i$ is a prime number.
\end{definition}
\begin{theorem}[Fundamental theorem of arithmetic, existence] \nirindex{Fundamental theorem of arithmetic} \label{thm:fta}
    Every natural number $n \geq 2$ has a prime factorization.
\end{theorem}
\begin{proof}
    Because this is a statement indexed by natural numbers, we will proceed by strong induction. Our base case is $n=2$, which is prime, so its prime factorization is just ``$2=2$.''

    For the inductive step, we may suppose $n>2$ and that all positive integers $k$ between $2$ and $n-1$ have prime factorizations. We now argue by cases: either $n$ is prime, or $n$ is not prime.
    \begin{itemize}
        \item If $n$ is prime, then like the $n=2$ case above, we see that ``$n=n$'' is our prime factorization.
        \item Otherwise, $n$ is not prime. However, $n>2$, so there are positive integers $a$ and $b$ such that $a,b>1$ while $n = ab$. Because $b>1$, we see $a<n$, and similarly $b<n$. Thus, by the inductive hypothesis, we see $a$ and $b$ both have prime factorizations, which we write as
        \[a=p_1\cdot p_2\cdot\ldots\cdot p_m\qquad\text{and}\qquad b=q_1\cdot q_2\cdot\ldots\cdot q_n.\]
        But $n=ab$, so we may write
        \[n=ab=(p_1\cdot p_2\cdot\ldots\cdot p_m)\cdot(q_1\cdot q_2\cdot\ldots\cdot q_n)\]
        to provide a prime factorization of $n$.
    \end{itemize}
    The above cases finish the inductive step.
\end{proof}
\begin{remark}
    This proof of the fundamental theorem of arithmetic actually gives instructions for explicitly writing down the prime factorization: just keep factoring until you eventually get prime factors. %Sometimes, however, we are less fortunate, and cannot produce a so-called \emph{constructive} proof (i.e. one which proves the existence of a particular object by explicitly building it and exhibiting that it has the desired properties).
\end{remark}
\begin{remark}
    Another induction is able to show that prime factorizations of positive integers are unique, up to permutation of the factors. This proof is a little more involved (what does ``up to permutation of the factors'' even mean?), so we have not assigned it as an exercise, but the interested reader should attempt a proof.
\end{remark}
\begin{exercise} \label{exe:fta-uses-strong-induction}
    Explain why the proof of \Cref{thm:fta} requires strong induction and could not have simply used regular induction. In other words, where did we use the strong induction hypothesis?
\end{exercise}
\begin{exe} \label{exe:fta-regular-induction}
    Despite \Cref{exe:fta-uses-strong-induction}, prove \Cref{thm:fta} by regular induction as follows. Imitating \Cref{exe:binet-regular-induction}, let $P(n)$ denote the assertion ``all positive integers $k\ge2$ such that $k\le n$ have a prime factorization.'' Prove $P(n)$ is true for all $n\ge2$ by regular induction.
\end{exe}
When using strong induction, be vigilant! It's easy to make silly mistakes if you don't complete the whole process of an inductive argument. Here's an example.
\begin{badtheorem}
    For all $n$, we have $\frac{d}{dx}(x^n)=0$.
\end{badtheorem}
\begin{proof}[Bad proof]
    Our base case is $n=0$, where $\frac{d}{dx}(1)=0$. For our inductive step, suppose that $\frac{d}{dx}(x^k)=0$ for $k<n$. Then, by the product rule, we compute
    \[ \frac d{dx}\left(x^{n+1}\right)= \frac d{dx}\left(x^n \cdot x^1\right)=x^n\cdot\frac d{dx}\left(x^1\right)+ x^1\cdot\frac d{dx}\left(x^n\right) =x^n\cdot0+x^1\cdot0=0,\]
    which finishes.
\end{proof}
So what went wrong? It turns out that while the above manipulation is valid for all $n\ge 1$, it isn't for $n=0$: because the inductive step breaks for $x^1$, this incorrect step allowed the rest to follow. In other words, the issue is that we did not prove the base case $n = 1$, but we assumed it was true when performing the inductive step.

\subsection{Well-Ordering}
We'll now turn our attention to the notion of ``well-ordering,'' which formalizes both forms of induction we've seen so far, and will allow us to easily deduce their equivalence. In addition, well-ordering will let us use induction on much more exotic sets than $\NN$.

Intuitively, a ``well-ordering'' is a total order with minimums. Let's make this precise.
% Let $X$ be a set. A \emph{total order} (or \emph{linear order}) on $X$ is a subset $S$ of $X \times X$ such that:
% \begin{enumerate}
%     \item For each $x, y \in X$ either $(x, y) \in S$ or $(y, x) \in S$ (or both).
%     \item If $(x, y) \in S$ and $(y, x) \in S$ then $x = y$.
%     \item If $(x, y) \in S$ and $(y, z) \in S$ then $(x, z) \in S$.
% \end{enumerate} If $X$ has a total ordering on it, then $X$ is said to be \emph{ordered} (or occasionally \emph{totally-ordered} or \emph{linearly-ordered})
% \end{definition} We will often denote an element $(x, y) \in S$ by $x \leq y$. This notation is not a coincidence; it is meant to be suggestive of the ordering you're used to on sets like $\NN$, $\ZZ$, or $\RR$. The intuition of a total order is that it generalizes the idea that elements of a set can be "compared" in a way which shares the same properties as these more familiar orderings.

% In fact, we've already seen a similar idea before, with the partial ordering $\subseteq$ (see the remark after theorem 2.8), which is a slightly weaker type of order --- it does not demand that every pair of subsets of a set can be compared. For instance, $\{1\}$ and $\{2\}$ are subsets of $\{1, 2\}$, but neither is a subset of the other. Both total and partial orders are special types of relations, which we will explore further in chapter 4.
\begin{definition}
    Let $\le$ be a total order on a set $X$, and let $S\subseteq X$ be a subset. An element $x \in S$ is \dfn{minimal} (in $S$) if and only if $x \le y$ for each $y \in S$. An element $x \in S$ is \dfn{maximal} (in $S$) if and only if $y \le x$ for each $y \in S$.
\end{definition}
\begin{exe}
    Let $\le$ be a total order on a set $X$, and let $S\subseteq X$ be a subset. Further, let $x,y\in S$.
    \begin{itemize}
        \item If $x$ and $y$ are minimal in $S$, then $x = y$.
        \item If $x$ and $y$ are maximal in $S$, then $x = y$.
    \end{itemize}
\end{exe}
We are now ready to define well-orders.
\begin{definition}
    A total order $\le$ on a set $X$ is a \dfn{well-ordering} if any nonempty subset $Y \subseteq X$ has a minimal element. In this case, we say that $X$ is \textit{well-ordered} under $\le$.
\end{definition}
\begin{ex} \label{ex:n-well-order}
    The set $\NN$ is well-ordered under its usual ordering. Intuitively, we can see this as follows: for any nonempty set $S$, find some element $s\in S$. Then the set
    \[S'\coloneqq S\cap\{0,1,2,\ldots,s\}\]
    is finite ($S'$ has at most $s+1$ elements) and nonempty ($S'$ contains $s$), so $S'$ has a minimal element. But the minimal element of $S\cap\{0,1,2,\ldots,s\}$ will also be minimal in $S$, so $S$ has a minimal element.
\end{ex}
\begin{exe}
    Is $\ZZ$ well-ordered under its usual ordering? If so, prove that it is. If not, can you come up with a different total order on $\ZZ$ under which it is well-ordered?
\end{exe}
There turns out to be a connection between well-ordering and induction. This is best seen by example. Let's redo the proof of \Cref{exe:perfect-square}.
\begin{example} \label{exe:perfect-square-wop}
    For each $n\in\NN$, we use the well-ordering of $\NN$ in order to show that the number
    \[S_n\coloneqq1+3+\cdots+(2n+1)\]
    is always a perfect square.
\end{example}
\begin{proof}
    In fact, we claim that $S_n=(n+1)^2$ for each $n$. To see this, we proceed by contradiction: suppose for the sake of contradiction that the set
    \[S\coloneqq\left\{n\in\NN:S_n\ne(n+1)^2\right\}\]
    is nonempty. By the well-ordering principle, we may find a minimal element $n_0\in S$. Then there are two cases, which correspond to the base case and the inductive step of our induction.
    \begin{enumerate}
        \item Base case: note that $n_0>0$ because $n_0=0$ has $S_n=1=(0+1)^2$, so $n_0\notin S$.
        \item Inductive step: because $n_0>0$, we see that $n_0-1>0$ and so $n_0-1\in\NN$. However, because $n_0$ is minimal, we see $n_0-1\in S$, so
        \[1+3+\cdots+(2(n_0-1)+1)=S_{n_0-1}=(n_0-1+1)^2=n_0^2.\]
        Adding $2n_0+1$ to both sides, we conclude $S_{n_0}=(n_0+1)^2$, so $n_0\notin S$. This is our contradiction.
        \qedhere
    \end{enumerate}
\end{proof}
Let's explain this connection more abstractly. In the following theorem, we abbreviate the statement that $x \leq y$ and $x \neq y$ by $x < y$ (which should hopefully agree with your intuition for strict inequalities).
\begin{theorem} \label{thm:well-order-is-induction}
    Let $X$ be a set, and let $\le$ be a total order on $X$. The following are equivalent.
    \begin{enumerate}[label=(\alph*)]
        \item The total order $\le$ is actually a well-ordering on $X$.
        \item Let $P$ be any statement indexed by $X$. Suppose that for every $x \in X$, if $P(y)$ is true for each $y \in X$ such that $y < x$, then $P(x)$ is true. Then $P(x)$ is true for all $x \in X$.
    \end{enumerate}
\end{theorem}
It is helpful to read the below proof imagining that we set $X=\NN$ the entire time.
\begin{proof}
    We have to show that (a) implies (b) and that (b) implies (a). Roughly speaking, the idea in this proof is to figure out how to translate between ``properties'' and ``sets.''
    \begin{itemize}
        \item We show (a) implies (b). The main idea is to construct a subset of $X$ that we can use the well-ordering on. Imitating the construction of \Cref{exe:perfect-square-wop}, we set
        \[S\coloneqq\{x\in X:P(x)\text{ is false}\}.\]
        If $S$ is nonempty, then $P(x)$ is true for all $x\in X$, so we are done.
    
        Thus, we suppose for the sake of contradiction that $S$ is nonempty. But $X$ is well-ordered! As such, we may let $s_0$ be the minimal element of $S$. However, if $y<y_0$, then $y$ cannot be in $S$ because $y_0$ is minimal in $S$, so $P(y)$ must be true. It follows by hypothesis on $P$ that $P(y_0)$ is true, so $y_0\notin S$. But $y_0\in S$ by construction, so this is our contradiction.

        \item We show (b) implies (a). We would like to show that all nonempty sets have a minimal element. This proof will use contraposition a few times, so pay attention. Arguing by contraposition, we show that any set $S$ without a minimal element must be empty.
        
        Reversing the previous proof, the main idea is to construct a property $P$ that we can use the hypothesis on. Thus, for $x\in X$, we let $P(x)$ be the property that ``$x\notin S$.'' We would like to show that $P(x)$ holds for all $x\in X$.

        Well, given any $x\in X$, we claim if $P(y)$ is true for each $y\in X$ with $y<x$, then $P(x)$ is true. This will finish by hypothesis. To see this, we again argue by contraposition: given that $P(x)$ is false, we need to show that there is some $y\in X$ with $y<x$ and $P(y)$ false.

        Translating, we would like to show that, if $x\in S$, then there is some $y\in X$ with $y<x$ and $y\in S$. However, this is exactly the statement that $x\in S$ is not a minimal element, which is true because $S$ has no minimal elements!
    \end{itemize}
    The above implications complete the proof.
\end{proof}

\subsection{Well-Ordering for \texorpdfstring{$\NN$}{N}}
In this section, we explain the somewhat cryptic comments on ``explanations'' of regular induction and strong induction. The issue here is that, when one wants to define the natural numbers $\NN$, one often just assumes that regular induction or strong induction or something similar holds. Thus, any ``proof'' of these will likely end up being rather circular.

With that said, one common way to define the natural numbers $\NN$ is by assuming that they are well-ordered, as we remarked in \Cref{ex:n-well-order}. We will take this approach.
\begin{axiom}[well-ordering principle] \nirindex{well-ordering principle} \label{ax:wop}
    The usual total ordering $\le$ on $\NN$ is a well-order.
\end{axiom}
% In other words, every subset $Y$ of $\NN$ has a least element $y$, smaller than every other element of $Y$. The least element of $\NN$ is $1$. The least element of $\{222, 6, 9, 15\}$ is $6$. The least element of $\{0, 1, 2, 3\}^c$ is $4$. And so on. As we will prove, whenever a set $X$ has a property similar to the well-ordering principle, we can induct on the subsets of $X$.
We call the Well-ordering principle an ``axiom'' to remind ourselves that it is not a theorem: it's part of the definition of $\NN$!
% there is no question of proving the well-ordering principle. It is a feature built into the logical system we are working in! (If you want to learn about such technicalities, you should take an upper-division class on logic, in particular Math 135: Set Theory.)
It is not so different from the axioms ``if $p$ is a sentence which is not false, then $p$ is true" or ``if $X$ is a nonempty set, then we can pick an arbitrary element of $X$" that we have taken for granted since day one.

\begin{exercise}
    Let $m$ and $n$ be positive integers. Show, using \Cref{ax:wop}, that there exist integers $q$ and $r$ with $0 \leq r < m$ such that
    \begin{equation*}
        n = qm + r
    \end{equation*}
\end{exercise}
\begin{theorem} \label{thm:well-order-equivs}
    The following are equivalent.
    \begin{enumerate}[label=(\alph*)]
        \item The well-ordering principle: any subset $S\subseteq\NN$ has a minimal element.
        \item Regular induction: let $P$ be any statement indexed by $\NN$ such that the following hold.
        \begin{enumerate}[label=\arabic*.]
            \item Base case: the statement $P(0)$ is true.
            \item Inductive step: for any $n\in\NN$, if $P(n)$ is true, then $P(n+1)$ is true.
        \end{enumerate}
        Then $P(n)$ is true for all $n\in\NN$.
        \item Strong induction: let $P$ be any statement indexed by $\NN$ such that the following hold for some $m\in\NN$.
        \begin{enumerate}[label=\arabic*.]
            \item Base case: the statements $P(0),P(1),\ldots,P(m)$ are all true.
            \item Inductive step: for any $n>m$, if $P(k)$ is true for all $k<n$, then $P(n)$ is true.
        \end{enumerate}
        Then $P(n)$ is true for all $n\in\NN$.
    \end{enumerate}
\end{theorem}
\begin{proof}
    By \Cref{thm:well-order-is-induction}, statement (a') is equivalent to the following more inductive statement.
    \begin{itemize}
        \item[(a')] Let $P$ be any statement indexed by $\NN$. Suppose that for every $x \in \NN$, if $P(y)$ is true for each $y \in \NN$ such that $y < x$, then $P(x)$ is true. Then $P(x)$ is true for all $x \in \NN$.
    \end{itemize}
    Now, to prove a theorem claiming more than two statements are equivalent, the most efficient approach is to typically show that each statement implies the next. In other words, we show (a') implies (b), that (b) implies (c), and that (c) implies (a'). Intuitively, we can see somewhat visually that (a') looks a lot like (c), so the portions of this proof to really pay attention to are (a') implies (b) and (b) implies (c).
    \begin{itemize}
        \item We show (a') implies (b). We expect strong induction (which (a') is similar to) to be ``stronger'' than regular induction, so this proof will be direct. Let $P$ be a statement satisfying the hypotheses of induction. We show that $P(n)$ is true for all $n\in\NN$ by using (a').

        Indeed, select some $x\in\NN$ such that $P(y)$ is true for each $y\in\NN$ with $y<x$. We need to show $P(x)$ is true. There are two cases.
        \begin{itemize}
            \item If $x=0$, then $P(0)$ is true by the base case of the induction.
            \item If $x>0$, then note $x-1\in\NN$ and $x-1<x$, so $P(x-1)$ is true by hypothesis on $x$. Thus, $P(x)$ is true by the inductive step.
        \end{itemize}
        The above checks show that the hypotheses of (a') holds for $P$, so $P(n)$ is true for all $n\in\NN$.
        
        \item We show (b) implies (c). This is the hardest part of the proof. Given our $P$ and $m\in\NN$ satisfying the hypotheses of (c), the main idea is to again to adjust $P$ to some $P'$ to apply regular induction. Imitating \Cref{exe:fta-regular-induction}, we let $P'(n)$ denote the assertion ``$P(k)$ holds for all $k\le n+m$.''

        We now show $P'(n)$ is true for all $n\in\NN$ by regular induction.
        \begin{enumerate}
            \item Base case: note $P'(0)$ is just the base case of (c).
            \item Inductive step: given $n\in\NN$ such that $P'(0)$ is true, we note the statements $P(0),P(1),\ldots,P(n+m)$ are all true. It follows by the inductive step of (c) that $P(n+m+1)$ is also true. So we see all statements $P(0),P(1),\ldots,P(n+m+1)$ are true, makaing $P'(n+1)$ true.
        \end{enumerate}
        It follows that $P'(n)$ is true for all $n\in\NN$ by regular induction.

        \item We show (c) implies (a'). Suppose $P$ satisfies the condition of (a'). Then we set $m\coloneqq0$ and apply strong induction on $P$ to show $P(n)$ for all $n\in\NN$.
        \begin{enumerate}
            \item Base case: we show $P(0)$ is true. Well, there are no $y\in\NN$ such that $y<0$, so $P(y)$ vacuously holds for all of them. We conclude that $P(0)$ is true.
            \item Inductive step: suppose $n>0$ and that $P(k)$ is true for all $k<n$. Then $P(n)$ is true directly from (a').
        \end{enumerate}
        We conclude that $P(n)$ is true for all $n\in\NN$ by strong induction.
    \end{itemize}
    The above implications complete the proof.
\end{proof}

% Let $P$ be a statement indexed by $\NN$.\\
% \indent (1 $\implies$ 2) Suppose that the well-ordering principle holds. Suppose further that we have shown that $P(0)$ is true, and that if $P(n)$ is true for some $n \in \NN$ then $P(n + 1)$ is true as well. Let $F$ be the subset of $\NN$ consisting of each $n \in \NN$ such that $P(n)$ is false, and suppose towards contradiction that $F$ is nonempty. Then by the well-ordering principle, $F$ has a minimal element $n_0$.\\
% \indent Clearly $n_0 \neq 0$, as we know that $P(0)$ is true, whereas $P(n_0)$ is false since $n_0 \in F$. But $0$ is minimal in $\NN$, so $n_0 > 0$, and therefore $n_0 - 1 \geq 0$. Now, since $n_0 - 1 < n_0$ and $n_0$ is minimal in $F$, we have $n_0 - 1 \in \NN\setminus F$, so $P(n_0 - 1)$ is true. But this implies that $P(n_0)$ is true by hypothesis, which is a contradiction, so $F$ must be empty, and therefore $P(n)$ is true for each $n \in \NN$. Thus the well-ordering principle implies regular induction.\\
% \indent (2 $\implies$ 3) Suppose that regular induction holds. Suppose further that we have shown that $P(0), P(1), \dots, P(m)$ are true for some nonnegative $m$, and that for each $j \in \NN$ if $P(k)$ is true for all $k < j$ then $P(j)$ is true. For each $n \in \NN$, let $Q(n)$ be the statement that $P(0), \dots, P(n)$ are true. We will use (weak) induction to show that $Q(n)$ is true for each $n \in \NN$.\\
% \indent Our base case is immediate, for $P(0)$ is true implies that $Q(0)$ is also true. Now, suppose $Q(n)$ is true for some $n \in \NN$. Then $P(0), \dots, P(n)$ are true, which implies that $P(n + 1)$ is true, and therefore $Q(n + 1)$ is true. Therefore by regular induction $Q(n)$ is true for each $n \in \NN$. But if $Q(n)$ is true then certainly $P(n)$ is as well, so $P(n)$ is true for each $n \in \NN$. Therefore regular induction implies strong induction.\\
% \indent (3 $\implies$ 1) Suppose that strong induction holds, and let $S$ be a nonempty subset of $\NN$. Suppose further that $S$ has no minimal element, let $T$ be the complement of $S$ in $\NN$, and for each $n \in \NN$ let $R(n)$ be the statement that $n \in T$. Then $R(0)$ is true, for otherwise $0 \in S$, in which case 0 is minimal in $S$ (as it is smaller than any other element of $\NN$).\\
% \indent Finally, let $j \in \NN$, and suppose $R(k)$ is true for each $k < j$. If $R(j)$ is false then $j \in S$, and since $R(k)$ is true for all $k$ smaller than $j$ we know that $S$ contains no element less than $j$, so $j$ is minimal in $S$. This is impossible, so $R(j)$ must be true. Hence, by strong induction $R(n)$ is true for each $n \in \NN$, or equivalently, $T = \NN$. It follows that $S$ is empty, so we conclude that every nonempty subset of $\NN$ has a minimal element. Thus strong induction implies well-ordering, completing the proof.

\subsection{Problems}
\begin{homework}
    Prove the following.
    \begin{enumerate}[label=(\alph*)]
        \item For any positive integer $n$, there exists an integer $k\ge0$ such that $2^k\le n<2^{k+1}$.
        \item Every positive integer $n$ can be written as the sum of distinct powers of $2$.
    \end{enumerate}
\end{homework}

\begin{homework}
    Deduce the well-ordering principle, \Cref{ax:wop}, from the more following plausible-sounding statement: for every $n \in \NN$, there are only finitely many $m \in \NN$ such that $m < n$.
\end{homework}

\begin{homework}
    Two positive integers $a$ and $b$ are said to be ``coprime'' if and only if they share no positive common divisors besides $1$. In other words, if $a$ and $b$ are both divisible by a positive integer $d$, then $d>0$. We will prove the following theorem.
    \begin{theorem}[Bezout]
        Let $x$ and $y$ be coprime nonzero integers. Then there exist integers $a$ and $b$ such that
        \begin{equation*}
            ax + by = 1
        \end{equation*}
    \end{theorem}
    Fill in the sketch below.
    \begin{enumerate}[label=(\alph*)]
        \item Let $S$ denote the set of integers of the form $ax+by$ where $a$ and $b$ are any integers. Show that $S$ contains a positive integer and thus a least positive integer $g$.
        \item Use the division algorithm to show that $g$ divides $x$. Similarly, show that $g$ divides $y$.
        \item Show that there exist integers $a$ and $b$ such that $ax+by=1$.
    \end{enumerate}
\end{homework}
\end{document}